% ======= PREAMBLE (clean) =======
\usepackage[T1]{fontenc}
\usepackage[utf8]{inputenc}

% Grafik & Layout
\usepackage{graphicx}
\usepackage{xcolor}
\usepackage{geometry}
\usepackage{fancyhdr}
\usepackage{titlesec}
\usepackage{tocloft}
\usepackage{lastpage}
\usepackage{adjustbox}
\usepackage{multicol}
\usepackage{array}
\usepackage{microtype}   % bessere Laufweiten/Umbrüche
\usepackage{ragged2e}    % erweiterte Absatz-Optionen (u.a. \justifying)
\usepackage{enumitem}    % straffe Listen
\usepackage{xurl}        % Umbrüche in URLs

% Roboto als Standard-Sans
\usepackage[sfdefault]{roboto}
\renewcommand{\familydefault}{\sfdefault} % alles Sans

% --- Farben ---
\definecolor{gtablue}{HTML}{0072B2}
\definecolor{gtagreen}{HTML}{28A745}
\definecolor{gtared}{HTML}{DC3545}
\definecolor{gtagray}{HTML}{666666}

% --- Seitenränder (Dokument-Default) ---
\geometry{
  top=25mm, bottom=25mm,
  left=25mm, right=25mm,
  headheight=10mm, headsep=6mm
}

% --- Header/Fuß ---
\pagestyle{fancy}
\fancyhf{}
\renewcommand{\headrulewidth}{0pt}
\renewcommand{\footrulewidth}{0pt}
\fancyhead[L]{\includegraphics[height=6mm]{aux/GTA_logo}} % Dateiendung weglassen, nutzt PDF wenn vorhanden
\fancyfoot[R]{\thepage}

% Leerer Stil für das Cover
\fancypagestyle{cover}{%
  \fancyhf{}
  \renewcommand{\headrulewidth}{0pt}
  \renewcommand{\footrulewidth}{0pt}
}

% --- Abschnitts-Typo (einheitlich, ohne Font-Switches) ---
\titleformat{\section}
  {\Large\bfseries\color{black}}{\thesection}{0.75em}{}
\titlespacing*{\section}{0pt}{*2}{0.6\baselineskip}

% --- Grundschrift & Abstände ---
\let\oldnormalsize\normalsize
\renewcommand{\normalsize}{\oldnormalsize\fontsize{11pt}{15pt}\selectfont}
\setlength{\parindent}{0pt}
\setlength{\parskip}{6pt plus 2pt minus 1pt}
\setlist[itemize]{leftmargin=*, itemsep=2pt, topsep=2pt}
\setlist[enumerate]{leftmargin=*, itemsep=2pt, topsep=2pt}

% --- TOC ---
\renewcommand{\cftsecfont}{\bfseries}
\renewcommand{\cftsecleader}{\cftdotfill{\cftdotsep}}
\renewcommand{\cftsecpagefont}{}
\setlength{\cftsecindent}{0pt}

% --- Grafiksuche ---
\graphicspath{{./}} % Logos und Bilder im Arbeitsverzeichnis

% --- Cover-Hilfsbefehle ---
\newcommand{\covertitlefont}{\fontsize{24}{28}\selectfont\bfseries}
\newcommand{\coversubfont}{\fontsize{12}{14}\selectfont}
\newcommand{\coverbody}{\small}

% Logo-Reihe: größere Logos wie im Original, mit Baseline-Ausrichtung
\newcommand{\logorow}{%
  \noindent
  \begin{minipage}[c]{0.3\textwidth}
    \centering
    \includegraphics[width=0.9\linewidth, keepaspectratio]{aux/GTA_logo}
  \end{minipage}%
  \hfill
  \begin{minipage}[c]{0.35\textwidth}
    \centering
    \includegraphics[width=1.1\linewidth, keepaspectratio]{aux/SECO_logo}
  \end{minipage}%
  \hfill
  \begin{minipage}[c]{0.3\textwidth}
    \centering
    \includegraphics[width=0.8\linewidth, keepaspectratio]{aux/C4TP_logo}
  \end{minipage}%
}

% Cover-Umgebung: eigene Ränder, kein Kopf/Fuß, linksbündig
\newenvironment{covertitlepage}
  {\begin{titlepage}\newgeometry{top=28mm,bottom=28mm,left=25mm,right=25mm}%
   \thispagestyle{cover}\sffamily\justifying\setlength{\parskip}{6pt}}
  {\restoregeometry\end{titlepage}}

% --- Hyperref: zuletzt laden ---
\usepackage{hyperref}
\hypersetup{
  colorlinks=true,
  linkcolor=black, urlcolor=gtablue, citecolor=black,
  pdftitle={Relative Trump Tariff Advantage: Chart Book},
  pdfauthor={Johannes Fritz}
}

% --- Absatz-Layout: akademischer Stil ---
% Vollausrichtung im gesamten Dokument und spuerbare Absatzabstaende
\AtBeginDocument{\justifying}

% Hyphenation policy: strongly discourage hyphenation globally
\hyphenpenalty=10000
\exhyphenpenalty=10000
% ======= END PREAMBLE =======